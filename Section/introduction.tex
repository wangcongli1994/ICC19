\section{Introduction}
HTTPS is becoming the de facto standard of secure web applications
 to provide  authentication, data confidentiality and integrity \cite{felt2017measuring}.
%It is used for hundreds of million of times every day in browsers .
HTTPS implements HTTP over TLS.
    In the TLS negotiation, a browser verifies the server certificate by building a certificate chain,
    and validates the certificate subject with the domain visited.
Recently,
    the security of HTTPS is further enhanced by HPKP that pins the public keys of the website in browsers \cite{Evans2015rfc7469},
     and HSTS that forces browsers to visit the website only using HTTPS but not HTTP \cite{hodges2012rfc6797}.

There are various errors in the TLS negotiations and the security-enhancements of HTTPS.
Browsers show security warnings to users with different levels, on different HTTPS errors:
(\emph{a}) a low-risk warning is shown in the address bar, no interrupting the browsing;
(\emph{b}) a medium-risk warning is shown as a full-screen page, interrupting the browsing,
    but the user may choose to ignore the warning by actively clicking to continue;
and (\emph{c}) a high-risk one is also shown as a full-screen page, stopping the browsing,
    but the user may choose to retry. % by clicking a button;
%and (\emph{d}) a fatal warning immediately close the connection and stop the browsing.

The security warnings aim to protect users to establish secure connections with the target website,
    but not a man-in-the-middle (MitM) or impersonation attacker.
However, an error does not definitely result from the attacks.
So, in order to balance security and usability,
    each browser implements its own design of security procedures,
        producing different behaviors and warnings on HTTPS errors.
This work presents a list of common HTTPS errors,
and investigates the browser behaviors on each error.
The error list covers certificate verification of basic fields and extensions,
        name validation, and HTTPS security-enhancement of HPKP and HSTS.

The evaluated browsers are Chrome, Firefox, Edge, and IE on Windows 10.
Our study
    discloses the browser defects on handling HTTPS errors in terms of cryptographic algorithm,
     certificate verification, %extension, certificate revocation,
      name validation,
      HPKP, and HSTS.
Some serious defects cause a forged or revoked certificate to be (probably) accepted by users:
\begin{itemize}
\item
Edge and IE consider MD5 as secure,
    so a server certificate signed or forged by this broken algorithm \cite{md5-forged-cert1, md5-forged-cert2}
     is accepted without warnings.
Meanwhile, all browsers show no warnings
    on certificates binding an RSA-1024 key pair,
        which is prohibited by the community \cite{CAB2018BR, NIST2005ECC}.
%Chrome, Firefox, Edge, IE: Signature Algorithm. The four browsers consider RSA1024 secure.
%Edge, IE: Digest Algorithm. Edge and IE consider secure.
\item
Chrome does not check the certificate revocation status through commonly-used CRL and OCSP,
    and Firefox does not support CRL. Thus,
    a revoked TLS server certificate may be accepted by Chrome and Firefox.
\item
Firefox does not process the Basic Constraints extension correctly.
Thus, a malicious server (but not an accredited CA) may sign TLS server certificates arbitrarily,
     and such forged certificate chains will be probably accepted by Firefox.
    % including the fields of cA and pathLenConstraint.
%Chrome, Edge, IE: CA flag. when the CA flag field of a server certificate is set to ��true��, Chrome, Edge and IE consider it secure.
%Firefox pathLenConstraint. when a subordinate CA does not comply with pathLenConstraint, Firefox consider it secure.
Chrome, %Firefox,
 Edge and IE do not process the extensions of Key Usage and Extended Key Usage correctly.
%even when these extensions appear inconsistently in certificates,    no warning shows.
So a certificate for other purposes may be maliciously used in HTTPS, and accepted by these browsers.
% key usage and extended key usage. Chrome, Edge, and IE don't display any warnings on certificates whose key usage and extended key usage is not consistent.
\item
In all browsers,
    the visits over plain HTTP that violates the HSTS policy,
        trigger only the low-risk warnings, not interrupting the user's browsing.
%: HPKP. Chrome and Firefox does not perform HPKP check when the certificate chain chains up to a root certificate installed by the user.
However,
    such visits deviate from the user's willing or probably result from the impersonation attacks.
\item
The HPKP support of Chrome and Firefox
    does not work well,
    when the certificate chain is verified by a user-installed root CA certificate.
So the security-enhancement of HPKP does not  always take effect as expected.
\end{itemize}
There are also defects on the problems of usability.
%\begin{itemize}
%\item
Chrome implements the procedure of name validation,
    only with the Subject Alternative Name (SAN) extension but not the commonName (CN) field,
        which is different from other browsers.
So a legitimate server certificate considered as secure by Firefox, Edge and IE,
    may trigger a warning in Chrome.
%Each browser has
%Name validation, different. The four browsers have inconsistencies in checking CN and SAN.
%\item
%\textcolor[rgb]{1.00,0.00,0.00}{Firefox does not support self-signed certificate as server certificate. when FireFox encounters a website using a self-signed certificate, the only way to make it work is adding a Security Exception for that particular website.}
%\end{itemize}

%The main process of web browser certificate validation is as follows:
%    The web server sends an endpoint certificate referencing its domain name,
%    as well as one or more intermediate certificates to a browser.
%    The browser must construct a trust chain from its trusted root certificates to the endpoint.
%    Certificate validation involves checking the whole chain for
%    that the visited domain name matches the subject in the certificate presented by the web server,
%    that the digital signature value is valid,
%    that all the certificates of the chain are within their validity period,
%    that the certificates haven��t been revoked, that various extensions are meet requirements,
%    and many other checks.
%    If that validation fails,
%    browsers display HTTPS security warnings to warn users of potential network attacks.
%    However, if warnings cannot communicate risks correctly, uses may make wrong decisions.


%Our work focuses on the X.509 certificate validation rather than the whole process of HTTPS implements,
%    but we hope our work can be useful for deploying HTTPS faster and better.

\noindent\textbf{Contribution.} We present a comprehensive list of HTTPS errors,
        and investigate the security warnings of four mainstream browsers on these errors.
Several defects are found,
    and these defects may result in the problems of security or usability in HTTPS.
We finally present some suggestions to improve the browsers' design of warnings.
%Based on the analysis results, we found that some browsers didn't handle properly,
%    for example,
%    firefox displays medium-risk warnings for manually installed self-signed certificates,
%    chrome displays secure for revoked certificates that only contain CRL information or OCSP information,
%    IE and Edge don��t provide any warning for MD5 and SHA-1 certificates,
%    and chrome provide medium-risk warnings for certificates whose SAN fields don't contain the domain we visited.
%    We accordingly make recommendations for browser warnings to convey appropriate levels of risk.
