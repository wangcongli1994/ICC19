\subsection{HTTPS Errors}
%    In this subsection,
%    we focus on various types of errors that may trigger browser security warnings.
We list
the errors of certificate verification of basic fields and extensions,
 name validation, and also security-enhancement in HTTPS.
These errors are induced in
    the evaluated browsers.

\subsubsection{Certificate Verification of Basic Fields}
In the TLS negotiation,
 a browser verifies the server certificate by building a valid certificate chain.
Each of the following errors leads to an invalid certificate chain.

%    Chain building Error occurs
%    if the certificate chain presented by a web server
%    cannot provide enough information for a browser to build a complete chain
%    from the endpoint certificate to a trusted root certificate in the browser��s root store.
\textbf{Untrusted root CA certificate.}
The root CA certificate of the certificate chain, sent by the visit server,
    is not trusted by browsers (i.e., not in the store of root CA certificates).
%    This error occurs when the chain of intermediate CAs linking the server certificate to a root CA untrusted by browsers.

\textbf{Untrusted self-signed server certificate.}
It is a special case of untrusted root CA certificates,
    when the visited server also acts as a root CA.
        %whose subject and issuer are the same.
The server certificate is self-signed,
    and not trusted by browsers.
%    Similarly, self-signed certificate is not trusted by browsers and must be installed manually.

\textbf{Incomplete certificate chain.}
Browsers cannot build a complete certificate chain,
    starting from a self-signed certificate to the received server certificate.
 %   a complete certificate chain cannot be formed due to the lack of intermediate certificates.
    %In order to build a complete certificate chain,
    %browsers obtain the intermediate certificates in three ways.
    %First, the web server presents its server certificate accompany with all the necessary intermediate certificates that can leads to a trusted root certificate.
    %Second, browsers reuse the cached intermediate certificates.
    %Third, browsers actively fetch the missing intermediate certificates through an AIA field.
    %An incomplete chain error occurs, when all the attempts fail.

\textbf{Expired server certificate.}
The certificate has expired.
    %A certificate is only valid between its Not Before date and Not After date.
    %Expired certificate means that the Not After date is earlier than current time.
    %An expired certificate error occurs if the Not After date is earlier than current time.

\textbf{Revoked server certificate.}
The received server certificate has been revoked.
The revocation status is distributed through CRL, OCSP, or CRLSet \cite{CRLSet}.
%        the CA claims that the server certificate is no longer valid before it expires by CRL, OCSP, or CRLset.

    %For security reasons,
    %CA can claim that a certificate is no longer valid before it expires by CRL, OCSP, and CRLset.
    %A revoked certificate error occurs when the server uses a revoked certificate.

\textbf{Weak signature algorithm.}
%    Weak signature algorithm refers to the unsafe, practically breakable algorithm, such as RSA512, RSA1024 \cite{CAB2018BR}, and ECC160 \cite{NIST2005ECC}.
The server certificate is signed by insecure signature algorithms,
    such as RSA-512, RSA-1024 \cite{CAB2018BR}, or ECC-160 \cite{NIST2005ECC}.
    %The CA/Browser Forum allows RSA and ECC keys in certificates.
    %All RSA keys should be at least 2048 bits \cite{CAB2018BR},
    %and ECC keys should be at least 256 bits \cite{NIST2005ECC}.
    %A weak signature algorithm error occurs when the key length does not meet the requirements.

\textbf{Weak hash algorithm.}
The certificate  is signed
    by insecure hash algorithms such as SHA-1 \cite{Google2017sha1} and MD5 \cite{wang2004collisions}.
    %According to the CA/Browser Forum,
    %supported digest algorithms are SHA-256, SHA-384, and SHA-512. The SHA-1 and MD5 digest algorithm have been shown to be insecure, such as two different PDF files with the same SHA-1 hash \cite{Google2017sha1}.

\subsubsection{Certificate Verification of Extensions}
The following certificate extensions are critical
        in certificate verification,
and errors may happen when browsers process these extensions.
    
%    In addition to the basic certificate validation, we also consider the certificate extension validation: Basic Constraints, Key Usage, and Extended Key Usage.

\textbf{Basic constraints.}
This extension consists two fields:
 cA and pathLenConstraint \cite{cooper2008rfc5280}.
The cA boolean indicates it is a CA certificate or not.
%    CA certificates must set the ``CA flag" as true, and server certificates must set the ``CA flag" as false.
The pathLenConstraint integer must appear on in a CA certificate,
    and 
    it gives
        the maximum number of intermediate CA certificates following this CA certificate in a valid certificate path.

\textbf{Key usage.}
%    ``no key usage" means that the certificate does not have a key usage field.
This extension defines the purpose(s) of the key pair bound in the certificate.
\textcolor[rgb]{1.00,0.00,0.00}{According to xxxx,
    a CA certificate must include the keyCertSign usage,
    and a TLS server certificate must include the keyEncipherment usage.}
    
%    ``server certificate: incorrect key usage" means that the server certificates don't include "Key Encipherment" in their key usage field.
 %%%% ʲô��׼Ҫ��key encipherment��Ҫ�У�������ȷ��������5280û�����Ҫ�󣡣�����
%    ``server certificate: correct key usage" means that the server certificates include "Key Encipherment" in their key usage field.
%    ``ca certificate: incorrect key usage" means that the CA certificates don't include "Certificate Signing" in their key usage field \cite{cooper2008rfc5280}.
    %The key usage extension indicates the purpose of the public key contained in a certificate.
    %According to RFC 5280 \cite{cooper2008rfc5280},
    %all CA certificates must include "Certificate Signing" in their key usage extension, and
    %all server certificates must include "Key Encipherment" in their key usage extension.

\textbf{Extended key usage.}
%    ``no extended key usage" means that the certificate does not have a extended key usage field.
This extension indicates one or more detailed purposes of the certified key pair,
    in addition to the basic purpose(s) in the Key Usage extension.
%    ``server certificate: incorrect extended key usage" means that the server certificates don't include ``serverAuth" in their extended key usage field.
%    ``server certificate: correct extended key usage" means that the server certificates include ``serverAuth" in their extended key usage field \cite{cooper2008rfc5280}.
\textcolor[rgb]{1.00,0.00,0.00}{A TLS server certificate must include the serverAuth extended key usage.}
    %Extended key usage extension further refines key usage extension.
    %If the extension is critical,
    %all server certificates must include "serverAuth" in their extended key usage extension.


\subsubsection{Name Validation}
    %Certificates are only valid for specific domain names.
Browsers check
    whether the certificate subject matches the domain visited or not,
    through the field of commonName and the extension of subject alternative name.
%    Name validation error means that domain names listed in a server certificate don't match the domain we visited.
There following errors of name validation may happen in commonName and/or subject alternative name.

%are many kinds of errors with respect to CN and SAN, for example, CN does not match the domain name visited, SAN does not match the domain name visited, and the lack of CN/SAN.

\textbf{Complete mismatch.}
It means that,
 there is no relation between the domain visited and the certificate subject.
    For example, a user visits www.foo.com but receives a certificate for www.bar.com.

\textbf{WWW mismatch.}
This error happens,
    when a user visits www.foo.com but receives a certificates for foo.com.

\textbf{Out-of-wildcard-scope subdomain.}
This error happens,
    when a user visits ab.xy.foo.com but receives a certificate for *.foo.com.

%\textbf{Common Name or Subject Alternative Name Mismatch.}
    %Common Name (CN) and Subject Alternative Name (SAN) fields indicate the certificate subject identity information, and they are usually domain names.

\subsubsection{HTTPS Security-Enhancement}
%\subsubsection{Certificate verification enhancement errors}
In recent years, several security-enhancements of HTTPS are proposed.
%    some security-enhancement policies have been added to HTTPS.
    We focus on HSTS and HPKP only in this paper.

\textbf{HPKP.}
%    HTTP Public Key Pinning (HPKP) \cite{Evans2015rfc7469} is used by web sites to instruct browsers to remember ("pin") the sites' public key over a period of time.
An HPKP error happens,
    when
    no public key in the received certificate chain %provided by the website
    matches the pinned public keys  \cite{Evans2015rfc7469},
        or    there is any error in the HPKP header.

\textbf{HSTS.}
%    HTTP Strict Transport Security (HSTS) \cite{hodges2012rfc6797} directs browsers to interact with given sites only over HTTPS.
% HSTS�Ľ�������related work���ɡ�
%
%    This policy is declared by web sites via the STS header field including three directives: max-age, includeSubDomains, and preload.
An HSTS error means that,
    browsers visit an HSTS website through HTTP \cite{hodges2012rfc6797},
    or there is any error in the HSTS header.
%     a website deploying HSTS has a certificate validation error 
