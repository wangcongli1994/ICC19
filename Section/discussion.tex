\section{Discussion and Suggestion}
%In the experiment, we found that different browsers have different processing methods for some unsafe scenarios.
%Through comparison between them, we have summarized the existing risks of several browsers.

\subsection{Certificate Revocation}
Most certificates in the Internet are revoked through CRL and/or OCSP,
    but CRL and OCSP are not supported by Chrome while Firefox does not support CRL.
The statistics in 2015  \cite{liu2015end} show that,
    99.9\% of server certificates contains a CRL distribution point, and
%        however, Chrome and Firefox don't display any warnings on revoked server certificates through CRL.
%        we think Chrome and Firefox need to be improved on this.
%
    95\% contains a OCSP responder.
%        however, Chrome does not display any warnings on revoked server certificates through OCSP.
%        we think Chrome need to be improved on this.

Chrome checks the certificate revocation status
    only through CRLSet, which only covers 0.35\% of all revocations in the Internet \cite{liu2015end}.
%        and Chrome only displays warnings on revoked certificates through CRLSet.
Although CRLset is constantly updated,
    in our tests,
     Chrome still consider the HTTPS visits as secure \textcolor[rgb]{1.00,0.00,0.00}{within a month} after the server certificate was revoked. %% ��û�и���׼ȷ��ʱ�䣿
This certificate is issued by Let's Encrypt and the serial number is 03e37c58091fb4056aa84130a1e54f8b58fd.
%        we think Chrome need to be improved on this.



\subsection{Weak Cryptographic Algorithm}
The following cryptographic algorithms are publicly prohibited or unrecommended \cite{CAB2018BR, NIST2005ECC},
but the processes of browsers can lead to (\emph{a}) forged certificates to be accepted
  or (\emph{b}) broken key pairs to be used in TLS negotiations.

The collisions of MD5 and SHA-1 have been found \cite{wang2004collisions, sha1-collision},
    and the weakness of MD5 was exploited to successfully forge certificates \cite{md5-forged-cert1, md5-forged-cert2}.
%Several security incidents have shown that,
%        attackers can use SHA-1 and MD5 to colliding fraudulent certificates,
Edge and IE do not show any warnings on certificates signed by MD5 or SHA-1.

1024-bit RSA is expected to be broken within 2015 to 2020 \cite{Berlin2017An},
    but none of these evaluated mainstream browsers show any warnings on certificates with 1024-bit RSA  key pairs.
The first factorization of 512-bit RSA was finished in 1999 \cite{cavallar2000factorization},
    and an 512-bit RSA key pair was broken in 73 days by a desktop computer in 2009 \cite{xxx}.
However, Chrome, Edge, and IE show only \emph{bypassable} Level-B warnings on 512-bit RSA.
On the contrary,
    these browsers show Level-D fatal warnings on a server certificate with 160-bit ECC key pair,
     which is stronger than 512-bit RSA \cite{ecc-vs-rsa}.

\subsection{Certificate Extension}
\textbf{Basic Constraint}
    We must strictly distinguish between server certificate and CA certificate.
    However, when the CA flag field of a server certificate is set to ``true",
        Chrome, Edge and IE think it safe.
    We think this three browsers need to be improved on this.

    In addition, if a CA has pathLenConstrait field, its subordinate CA should comply with this rule.
    However, Firefox does not display any warnings on pathLenConstrait error.
    We think Firefox need to be improved on this.

\textbf{Extended Key Usage and Key Usage}
According to RFC 5280, if a certificate contains both a key usage extension and an extended key usage extension,
    then both extensions MUST be processed independently
    and the certificate MUST only be used for a purpose consistent with both extensions.
    If there is no purpose consistent with both extensions, then the certificate MUST NOT be used for any purpose \cite{cooper2008rfc5280}.

For key usage and extended key usage of server certificate, there are two incorrect browser security warnings.
    First, Chrome, Edge, and IE does not display any warning on error of a server certificate that does not have extended key usage and has a incorrect key usage.
    Second, all the four browsers does not display any warning on error of a server certificate that with a correct extended key usage and a incorrect key usage.

% If a certificate contains both a key usage extension and an extended
%   key usage extension, then both extensions MUST be processed
 %  independently and the certificate MUST only be used for a purpose
  % consistent with both extensions.  If there is no purpose consistent
   %with both extensions, then the certificate MUST NOT be used for any
   %purpose.(RFC 5280)
\subsection{Name Validation}
The CA/Browser Forum suggests that
    deprecates the CN field and lists applicable names in SAN field.
However, The four browsers have inconsistencies in checking CN and SAN.
When visiting a domain whose domain name is included in the CN field and it does not have SAN field,
Chrome displays a Level-B medium-risk warning, but Firefox, Edge, and IE consider it secure.
This may lead to a problem of usability.

All browsers show Level-B warnings on different name validation errors:
complete mismatch,
WWW mismatch and out-of-wildcard-scope subdomain.
However,
    in our opinion,
    these errors bring different levels of risk
    and complete mismatch shall trigger a higher-level warning.



\subsection{HPKP}
    Of the four browsers, only Chrome and Firefox support HPKP.
    Chrome and Firefox does not perform HPKP check when the certificate chain chains up to a root certificate installed by the user.
    Thus, If a malware installs its root certificate on the user's computer,
        it may defeat the protections of HPKP and successfully launch the man-in-the-middle attack.
    We think Chrome and Firefox need to be improved on this.

%To enable certificate chain validation, Chrome has access to two stores of trust anchors: certificates that are empowered as issuers. One trust anchor store is the system or public trust anchor store, and the other other is the local or private trust anchor store. The public store is provided as part of the operating system, and intended to authenticate public internet servers. The private store contains certificates installed by the user or the administrator of the client machine. Private intranet servers should authenticate themselves with certificates issued by a private trust anchor.

%Chrome does not perform pin validation when the certificate chain chains up to a private trust anchor. A key result of this policy is that private trust anchors can be used to proxy (or MITM) connections, even to pinned sites. ��Data loss prevention�� appliances, firewalls, content filters, and malware can use this feature to defeat the protections of key pinning.
% We deem this acceptable because the proxy or MITM can only be effective if the client machine has already been configured to trust the proxy��s issuing certificate �� that is, the client is already under the control of the person who controls the proxy (e.g. the enterprise��s IT administrator). If the client does not trust the private trust anchor, the proxy��s attempt to mediate the connection will fail as it should.

%In order to control the browsing behavior of the internal network, some companies often pre-set the self-signed certificate in the computer in the office area, and replace the HTTPS connection certificate in the gateway to obtain the information contained in the HTTPS traffic, detect and prevent the information leakage inside the company, which is a typical MITM attack. Some malware may also use the same way to attack a personal computer. Many computer applications will trigger the UAC of the Windows system during installation. After clicking and agreeing, the computer settings can be modified. After the malware spoofs the user's consent, you can modify the computer settings, such as burying your own root certificate. Eavesdrop on all traffic through a self-signed certificate and send sensitive information to the attacker.





