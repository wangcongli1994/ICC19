\section{Discussion and Suggestion}
In the experiment, we found that different browsers have different processing methods for some unsafe scenarios. Through comparison between them, we have summarized the existing risks of several browsers.

\subsection{Weak Algorithm}
    Several security incidents have shown that,
        attackers can use SHA-1 and MD5 to colliding fraudulent certificates,
    but Edge and IE don't display any warnings on certificates with SHA-1/MD5 signatures.
    Edge and IE need to be improved on this.

    RSA512 have been broken since 2009 \cite{cavallar2000factorization}, but Chrome, Edge, and IE display only B-level (Medium-risk) warnings, we think the risk level of RSA512 is C-level (High-risk).

    RSA1024 is expectable to be broken within 2015 to 2020 \cite{Berlin2017An},
    but all the four browsers don't display any warnings on certificates with RSA1024 keys.
    They need to be improved on this.


%%%%%%%%%%%%%%%%%%%%%%%%%%%%
\subsection{Self-signed Server Certificate}
    Users can install self-signed server certificates to OS root store, Chrome,
        Edge, and IE will treat them secure.
    However, when FireFox encounters a website using a self-signed certificate,
        the only way to make it work is adding a Security Exception for that particular website.
    We think Firefox need to be improved on this.
%For a self-signed server certificate, it is both a root certificate and a CA certificate. This type of certificate can be added to the "Trusted Root Certification Authorities" store on Windows systems and made available to Chrome and Edge browsers for secure access. Firefox has a self-built certificate manager, but the manager does not store the server certificate in the certificate store. That is, if you need Firefox to trust a self-signed certificate, you must sign a server certificate by using the CA certificate, and install the root CA certificate into the certificate manager of the Firefox browser.

%When FireFox encounters a self-signed cert, it won't load the page. The way to make it work is to add a Security Exception for that particular website.

%The NSS root certificate store

\subsection{Revoked Server Certificate}
    99.9\% of the server certificates contains a CRL distribution point \cite{liu2015end},
        however, Chrome and Firefox don't display any warnings on revoked server certificates through CRL.
        we think Chrome and Firefox need to be improved on this.

    95\% of the server certificates contains a OCSP responder \cite{liu2015end},
        however, Chrome does not display any warnings on revoked server certificates through OCSP.
        we think Chrome need to be improved on this.

    CRLSet only covers 0.35\% of all revocations \cite{liu2015end},
        and Chrome only displays warnings on revoked certificates through CRLSet.
        Although CRLset is constantly updated, Chrome still consider my website is secure within a month after my server certificate was revoked.
        we think Chrome need to be improved on this.

\subsection{key usage and extended key usage}

According to RFC 5280, if a certificate contains both a key usage extension and an extended key usage extension,
    then both extensions MUST be processed independently
    and the certificate MUST only be used for a purpose consistent with both extensions.
    If there is no purpose consistent with both extensions, then the certificate MUST NOT be used for any purpose \cite{cooper2008rfc5280}.

For key usage and extended key usage of server certificate, there are two incorrect browser security warnings.
    First, Chrome, Edge, and IE does not display any warning on error of a server certificate that does not have extended key usage and has a incorrect key usage.
    Second, all the four browsers does not display any warning on error of a server certificate that with a correct extended key usage and a incorrect key usage.

% If a certificate contains both a key usage extension and an extended
%   key usage extension, then both extensions MUST be processed
 %  independently and the certificate MUST only be used for a purpose
  % consistent with both extensions.  If there is no purpose consistent
   %with both extensions, then the certificate MUST NOT be used for any
   %purpose.(RFC 5280)

\subsection{HPKP does not check the self-signed certificate chain}

    Based on the experimental results, we arrive at the the following conclusions.
    First, Of the four browsers, only chrome support HTTP Public Key is fixed (HPKP)
    Of all known GTPases, only eight are conserved across all three domains of life.
    Of the four browsers, only chrome supports HPKP
    if the server certificate chains to a manually installed root certificate,


    Any HPKP error does not trigger a warning
    the HPKP policy error warning will not be triggered even if the certificate used does not comply with the HPKP policy.

    Not only that, but this connection will also display the correct encryption and the correct icon.In other words, self-signed certificates are not pinned[].
%https://groups.google.com/a/chromium.org/forum/#!msg/blink-dev/he9tr7p3rZ8/eNMwKPmUBAAJ

In order to control the browsing behavior of the internal network, some companies often pre-set the self-signed certificate in the computer in the office area, and replace the HTTPS connection certificate in the gateway to obtain the information contained in the HTTPS traffic, detect and prevent the information leakage inside the company, which is a typical MITM attack. Some malware may also use the same way to attack a personal computer. Many computer applications will trigger the UAC of the Windows system during installation. After clicking and agreeing, the computer settings can be modified. After the malware spoofs the user's consent, you can modify the computer settings, such as burying your own root certificate. Eavesdrop on all traffic through a self-signed certificate and send sensitive information to the attacker.


