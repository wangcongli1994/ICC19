\section{Related work}

We are not aware of any prior work on discovery of risk level vulnerabilities in the browser HTTPS security warnings.
    However, previous studies have shown the need for that research to improve HTTPS implementations and browser warnings.

%��һ�ν�Invalid Certificates.
Researchers take large scale scans of HTTPS certificate ecosystem including IPv4 space, certificate transparency logs, and Alexa Top 1 Million websites
    to provide data sets of known certificates \cite{Durumeric2014Analysis} \cite{Durumeric2013ZMap} \cite{Holz2011The} \cite{Vandersloot2016Towards}.
    Using collections of certificates, researchers build certificate linters to measure certificate compliance with community standards such as CA/Browser Forum Baseline Requirements, RFC 5280 \cite{x509lint} \cite{certlint} \cite{kumar2018tracking}.
    As a result, they find a large proportion of invalid certificate in the wild.%����֤�鳷����֤����ڣ����㷨
    More than that, there are even many invalid certificates signed by browser-trusted CAs.
%��һ�ν�FrankenCerts
Other than large scale scans of HTTPS ecosystem,
    some researchers focus on measuring the quality of certificate validation in SSL/TLS implementations.
    Frankencert \cite{brubaker2014using} and Symcerts \cite{chau2017symcerts} are both designed to check whether SSL/TLS clients correctly validate X.509 certificates.
    RFCcert tests bugs in certificate validation of SSL/TLS implements by extracting rules directly from RFCs \cite{chen2018rfc}.

%��һ�ν� HPKP  HSTS
A number of HTTPS security polices have been developed to enhance HTTPS security.
    As a result, many studies focus on the development of HTTPS security polices.
    Clark et al. \cite{Clark2013SoK} provide a theoretical evaluation of HTTPS policies for improve the HTTPS security.
    Amann et al. \cite{Amann2017Mission} analyze the deployment of HSTS and HPKP by active scans and passive monitoring.
    Pokeinthe \cite{pokeinthe2017HSTS} ran a scan of the Alexa Top 1 Million websites in Q2 2017. The results shows a more than 40 percent increase in the use of HSTS preloded and HPKP.
    Santos et al. \cite{Santos2016Implementation} measure the implementation state of HSTS and HPKP in both browsers and servers.
    Across these early studies, we find that, although HSTS and HPKP are growing in adoption, the deployment is disappointing,
    misconfigurations may higher your risk.

%�⼸�ν�������ࡢ��������漶��
The browser security warning page is a security warning mechanism that many browsers have.
    When the browser finds that there is a risk associated with the page that the user is trying to browse,
    the page will appear.
    "Dissuading" users should not visit the website,
    but This warning can also be ignored if the user insists on access and is willing to take the risk.

We look at the classification process that the browser obtains the certificate error from the server.
    The similar classification error is also the same logic in the processing method,
    and the code part is also closely related.
    By clearly showing the type of error, you can effectively help website maintainers troubleshoot errors and improve the security of your website.

In 2013, Akhawe's study divided browser warnings into HTTPS warnings and phishing scam warnings,
    and studied the probability of users ignoring warnings and risking to continue browsing \cite{Akhawe2013Alice}.
    The 2017 study by Acer et al. further subdivided browser error warnings into server-side errors, client-side errors, and network errors,
    and counted the number of errors \cite{Acer2017Where}.
    Bernhard Amann's research proposes three categories: certificate chain construction error, certificate chain verification error and domain name verification error, and gives suggestions for reducing the probability of error \cite{Akhawe2013Here}.

None of the previous researches focused on the level of browser warnings.We built a warning level system and matched the HTTPS errors to the warning level.
