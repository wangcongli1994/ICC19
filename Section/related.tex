\section{Related Work}
%Previous studies have shown the need for analyzing browser security warnings of HTTPS errors.

%��һ�ν�Invalid Certificates.
Researchers take large scale scans of HTTPS certificate ecosystem including IPv4 space, certificate transparency logs, and Alexa Top 1 Million websites
    to provide data sets of known certificates \cite{Durumeric2014Analysis, Durumeric2013ZMap, Holz2011The, Vandersloot2016Towards}.
%    Using collections of certificates,
    researchers build tools to measure certificate compliance with
    community standards such as CA/Browser Forum Baseline Requirements, and RFC 5280.\cite{x509lint, certlint, kumar2018tracking}.
    As a result, they find a large proportion of invalid certificate in the wild.%����֤�鳷����֤����ڣ����㷨
    More than that, there are even many invalid certificates signed by browser-trusted CAs.

%��һ�ν�������ࡢ��������漶��
In 2013, Akhawe's study divided browser warnings into malware, phishing, and SSL warnings,
    and studied the probability of users ignoring warnings \cite{Akhawe2013Alice}.
    The 2017 study by Acer et al. further subdivided browser error warnings into server-side errors, client-side errors, and network errors,
    and counted the number of errors \cite{Acer2017Where}.
    Bernhard Amann's research proposes three categories: certificate chain construction error, certificate chain verification error and domain name verification error, and gives suggestions for reducing the probability of error \cite{Akhawe2013Here}.

%��һ�ν�FrankenCerts
%Other than large scale scans of HTTPS ecosystem,
    Some researchers focus on measuring the quality of certificate validation in SSL/TLS implementations.
    Frankencert \cite{brubaker2014using} and Symcerts \cite{chau2017symcerts} are both designed to check whether SSL/TLS clients correctly validate X.509 certificates.
    RFCcert tests bugs in certificate validation of SSL/TLS implements by extracting rules directly from RFCs \cite{chen2018rfc}.

%��һ�ν� HPKP  HSTS
A number of HTTPS security polices have been developed to enhance HTTPS security.
    As a result, many studies focus on the development of HTTPS security polices.
    Clark et al. \cite{Clark2013SoK} provide a theoretical evaluation of HTTPS policies for improve the HTTPS security.
    Amann et al. \cite{Amann2017Mission} analyze the deployment of HSTS and HPKP.
    Pokeinthe \cite{pokeinthe2017HSTS} ran a scan of the Alexa Top 1 Million websites in Q2 2017. The results shows a more than 40 percent increase in the use of HSTS preloded and HPKP in a year.
    Santos et al. \cite{Santos2016Implementation} measure the implementation state of HSTS and HPKP in both browsers and servers.
%    Across these early studies, we find that, although HSTS and HPKP are growing in adoption, the deployment is disappointing,  misconfigurations may higher your risk.


%None of the previous researches focused on the level of browser warnings.We built a warning level system and matched the HTTPS errors to the warning level.
