\section{Related Work}
%Previous studies have shown the need for analyzing browser security warnings of HTTPS errors.
ZMap \cite{Durumeric2013ZMap} conducted comprehensive scans of port 443 in the Internet and retrieve TLS certificates
    and \cite{DBLP:conf/uss/FeltBKPBT17} measured HTTPS adoption on the web
    from the perspectives of both users web developers in 2017.
TLS server certificates are collected
     to present the status of the deployed X.509 PKI \cite{Durumeric2014Analysis, Holz2011The}.


%��һ�ν�Invalid Certificates.
%Researchers take large scale scans of HTTPS certificate ecosystem including IPv4 space, certificate transparency logs, and Alexa Top 1 Million websites
 %   to provide data sets of known certificates \cite{Durumeric2014Analysis, Durumeric2013ZMap, Holz2011The, Vandersloot2016Towards}.
%    Using collections of certificates,

%researchers build tools to measure certificate compliance with
 %   community standards such as CA/Browser Forum Baseline Requirements, and RFC 5280.\cite{x509lint, certlint, kumar2018tracking}.
  %  As a result, they find a large proportion of invalid certificate in the wild.%����֤�鳷����֤����ڣ����㷨
   % More than that, there are even many invalid certificates signed by browser-trusted CAs.
The surveys of the HTTPS certificate ecosystem \cite{kumar2018tracking, Chung2016Measuring, Durumeric2014Analysis, Holz2011The}
    uncover a large proportion of invalid certificate chains in the wild.
%��һ�ν�FrankenCerts
%Other than large scale scans of HTTPS ecosystem,
Frankencerts \cite{brubaker2014using} and SymCerts \cite{chau2017symcerts} 
    analyzed the implementations of X.509 certificate validation in TLS software,
    by automated adversarial testing and symbolic execution,
while RFCcert \cite{chen2018rfc} finished the differential testing of certificate validation by extracting rules from RFCs.
These works disclose the bugs of certificate validation in popular SSL/TLS libraries, but not the browser defects in our work.

%��һ�ν�������ࡢ��������漶��
Akhawe et al. studied the browser warnings of malware, phishing and TLS error,
    and analyzed the probabilities that the warnings are ignored by users \cite{Akhawe2013Alice}.
\cite{Acer2017Where}    investigated the root causes of HTTPS certificate errors in the field,
    and found that a wide range of non-attack circumstances trigger browser warnings.
Akhawe et al. \cite{Akhawe2013Here} studied the TLS warnings based on a large-scale measurement of user data,
    and identified the low-risk scenarios that consume a large chunk of the user attention by using internal browser code.
Compared with these studies focusing on the TLS warnings the wild,
    we attempt to find the browser defects on handling HTTPS errors,
     which may cause a forged or revoked certificate to be accepted or the security-enhancement to be ineffective.
Besides,
    they do not cover the warnings on HTTPS security-enhancement errors.

%    Bernhard Amann's research proposes three categories: certificate chain construction error, certificate chain verification error and domain name verification error, and gives suggestions for reducing the probability of error \cite{Akhawe2013Here}.

%��һ�ν� HPKP  HSTS
Clark et al. \cite{Clark2013SoK} provide a comprehensive survey of HTTPS security-enhancements.
%A number of HTTPS security polices have been developed to enhance HTTPS security.
%    As a result, many studies focus on the development of HTTPS security polices.
    Amann et al. \cite{Amann2017Mission} analyze the deployment of HSTS and HPKP in web servers,
    %Pokeinthe \cite{pokeinthe2017HSTS} ran a scan of the Alexa Top 1 Million websites in Q2 2017. The results shows a more than 40 percent increase in the use of HSTS preloded and HPKP in a year.
   while Santos et al. \cite{Santos2016Implementation} measure the implementation state of HSTS and HPKP in both browsers and servers.
%    Across these early studies, we find that, although HSTS and HPKP are growing in adoption, the deployment is disappointing,  misconfigurations may higher your risk.


%None of the previous researches focused on the level of browser warnings.We built a warning level system and matched the HTTPS errors to the warning level.
